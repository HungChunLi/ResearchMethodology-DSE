% \newpage

\section*{表格}

\newpage
\begin{table}[H]
    \caption{AIDS及LAAIDS模型估計結果與顯著性}
    \centering
    \begin{tabular}{cc} % 建立兩欄表格佈局
        % 左欄上方表格
        \begin{subtable}[t]{0.45\textwidth}
            % \centering
            \begin{center}
                \caption{Alpha參數} \label{coef_alpha}
                % latex table generated in R 4.3.1 by xtable 1.8-4 package
% Mon Nov 25 17:19:36 2024
\begin{tabular}{lll}
  \hline
變數 & (1) & (2) \\ 
  \hline
alpha 1 & 1.238*** & 1.164*** \\ 
   & (0.085) & (0.079) \\ 
  alpha 2 & 0.537*** & 0.418*** \\ 
   & (0.129) & (0.116) \\ 
  alpha 3 & -0.300*** & -0.269*** \\ 
   & (0.074) & (0.063) \\ 
  alpha 4 & 0.694*** & 0.760*** \\ 
   & (0.063) & (0.056) \\ 
  alpha 5 & -1.169*** & -1.073*** \\ 
   & (0.220) & (0.192) \\ 
   \hline
\end{tabular}

                \vspace{0.5cm} % 增加表格間距
                \caption{Beta參數} \label{coef_beta}
                % latex table generated in R 4.3.1 by xtable 1.8-4 package
% Mon Nov 25 17:19:36 2024
\begin{tabular}{lll}
  \hline
變數 & (1) & (2) \\ 
  \hline
beta 1 & -0.091*** & -0.085*** \\ 
   & (0.007) & (0.007) \\ 
  beta 2 & -0.026* & -0.017. \\ 
   & (0.011) & (0.010) \\ 
  beta 3 & 0.030*** & 0.028*** \\ 
   & (0.006) & (0.005) \\ 
  beta 4 & -0.060*** & -0.065*** \\ 
   & (0.005) & (0.005) \\ 
  beta 5 & 0.147*** & 0.139*** \\ 
   & (0.019) & (0.016) \\ 
   \hline
\end{tabular}

            \end{center}
            \vspace*{1cm}
            \begin{singlespace}
                \begin{footnotesize}
                    \raggedright % 讓以下文字靠左對齊
                    \noindent {\it Notes:} 此表展示 AIDS 模型中各項參數的估計值及其顯著性檢驗結果。如 alpha 1 的估計值為 1.237,表明該參數對模型中需求分配的影響為正,且數值較大,標準誤為 0.085,顯示估計結果穩定。所有參數的 p 值均遠小於 0.05,表示這些參數在統計上顯著,同時也可看到 AIDS 模型在解釋台灣飲料市場需求方面具有良好表現。
                \end{footnotesize}
            \end{singlespace}
        \end{subtable} &

        % 右欄表格
        \begin{subtable}[t]{0.45\textwidth}
            \centering
            \footnotesize
            \caption{AIDS及LAAIDS模型估計結果與顯著性} \label{coef_gamma}
            % latex table generated in R 4.3.1 by xtable 1.8-4 package
% Mon Nov 25 17:19:36 2024
\begin{tabular}{lll}
  \hline
變數 & (1) & (2) \\ 
  \hline
gamma 1 1 & -0.007 & 0.081*** \\ 
   & (0.025) & (0.018) \\ 
  gamma 1 2 & -0.137*** & -0.112*** \\ 
   & (0.023) & (0.017) \\ 
  gamma 1 3 & 0.017. & -0.015* \\ 
   & (0.010) & (0.007) \\ 
  gamma 1 4 & -0.024 & 0.031* \\ 
   & (0.020) & (0.013) \\ 
  gamma 1 5 & 0.152*** & 0.015 \\ 
   & (0.031) & (0.017) \\ 
  gamma 2 1 & -0.137*** & -0.112*** \\ 
   & (0.023) & (0.017) \\ 
  gamma 2 2 & -0.111** & -0.119*** \\ 
   & (0.036) & (0.028) \\ 
  gamma 2 3 & -0.052*** & -0.060*** \\ 
   & (0.012) & (0.009) \\ 
  gamma 2 4 & -0.008 & 0.017 \\ 
   & (0.026) & (0.017) \\ 
  gamma 2 5 & 0.308*** & 0.274*** \\ 
   & (0.030) & (0.024) \\ 
  gamma 3 1 & 0.017. & -0.015* \\ 
   & (0.010) & (0.007) \\ 
  gamma 3 2 & -0.052*** & -0.060*** \\ 
   & (0.012) & (0.009) \\ 
  gamma 3 3 & -0.050*** & -0.037*** \\ 
   & (0.007) & (0.005) \\ 
  gamma 3 4 & 0.060*** & 0.039*** \\ 
   & (0.008) & (0.006) \\ 
  gamma 3 5 & 0.025. & 0.074*** \\ 
   & (0.014) & (0.011) \\ 
  gamma 4 1 & -0.024 & 0.031* \\ 
   & (0.020) & (0.013) \\ 
  gamma 4 2 & -0.008 & 0.017 \\ 
   & (0.026) & (0.017) \\ 
  gamma 4 3 & 0.060*** & 0.039*** \\ 
   & (0.008) & (0.006) \\ 
  gamma 4 4 & 0.135*** & 0.170*** \\ 
   & (0.029) & (0.019) \\ 
  gamma 4 5 & -0.163*** & -0.257*** \\ 
   & (0.020) & (0.012) \\ 
  gamma 5 1 & 0.152*** & 0.015 \\ 
   & (0.031) & (0.017) \\ 
  gamma 5 2 & 0.308*** & 0.274*** \\ 
   & (0.030) & (0.024) \\ 
  gamma 5 3 & 0.025. & 0.074*** \\ 
   & (0.014) & (0.011) \\ 
  gamma 5 4 & -0.163*** & -0.257*** \\ 
   & (0.020) & (0.012) \\ 
  gamma 5 5 & -0.322*** & -0.106** \\ 
   & (0.062) & (0.037) \\ 
   \hline
\end{tabular}

        \end{subtable} \\ % 換行分隔上下表格
    \end{tabular}
\end{table}


% \begin{table}[H]
%     \caption{AIDS模型估計結果與顯著性} \label{aids_coef}
%     \center
%     % latex table generated in R 4.3.1 by xtable 1.8-4 package
% Wed Nov 20 03:02:18 2024
\begin{tabular}{ccccc}
  \hline
 & Estimate & Std. Error & t value & Pr($>$$|$t$|$) \\ 
  \hline
alpha 1 & 1.24 & 0.09 & 14.51 & 0.00 \\ 
  alpha 2 & 0.54 & 0.13 & 4.17 & 0.00 \\ 
  alpha 3 & -0.30 & 0.07 & -4.06 & 0.00 \\ 
  alpha 4 & 0.69 & 0.06 & 10.93 & 0.00 \\ 
  alpha 5 & -1.17 & 0.22 & -5.31 & 0.00 \\ 
  beta 1 & -0.09 & 0.01 & -12.71 & 0.00 \\ 
  beta 2 & -0.03 & 0.01 & -2.43 & 0.02 \\ 
  beta 3 & 0.03 & 0.01 & 4.84 & 0.00 \\ 
  beta 4 & -0.06 & 0.01 & -11.58 & 0.00 \\ 
  beta 5 & 0.15 & 0.02 & 7.93 & 0.00 \\ 
  gamma 1 1 & -0.01 & 0.02 & -0.27 & 0.79 \\ 
  gamma 1 2 & -0.14 & 0.02 & -5.95 & 0.00 \\ 
  gamma 1 3 & 0.02 & 0.01 & 1.67 & 0.10 \\ 
  gamma 1 4 & -0.02 & 0.02 & -1.24 & 0.21 \\ 
  gamma 1 5 & 0.15 & 0.03 & 4.96 & 0.00 \\ 
  gamma 2 1 & -0.14 & 0.02 & -5.95 & 0.00 \\ 
  gamma 2 2 & -0.11 & 0.04 & -3.09 & 0.00 \\ 
  gamma 2 3 & -0.05 & 0.01 & -4.43 & 0.00 \\ 
  gamma 2 4 & -0.01 & 0.03 & -0.31 & 0.75 \\ 
  gamma 2 5 & 0.31 & 0.03 & 10.17 & 0.00 \\ 
  gamma 3 1 & 0.02 & 0.01 & 1.67 & 0.10 \\ 
  gamma 3 2 & -0.05 & 0.01 & -4.43 & 0.00 \\ 
  gamma 3 3 & -0.05 & 0.01 & -6.87 & 0.00 \\ 
  gamma 3 4 & 0.06 & 0.01 & 7.20 & 0.00 \\ 
  gamma 3 5 & 0.03 & 0.01 & 1.78 & 0.08 \\ 
  gamma 4 1 & -0.02 & 0.02 & -1.24 & 0.21 \\ 
  gamma 4 2 & -0.01 & 0.03 & -0.31 & 0.75 \\ 
  gamma 4 3 & 0.06 & 0.01 & 7.20 & 0.00 \\ 
  gamma 4 4 & 0.14 & 0.03 & 4.61 & 0.00 \\ 
  gamma 4 5 & -0.16 & 0.02 & -8.35 & 0.00 \\ 
  gamma 5 1 & 0.15 & 0.03 & 4.96 & 0.00 \\ 
  gamma 5 2 & 0.31 & 0.03 & 10.17 & 0.00 \\ 
  gamma 5 3 & 0.03 & 0.01 & 1.78 & 0.08 \\ 
  gamma 5 4 & -0.16 & 0.02 & -8.35 & 0.00 \\ 
  gamma 5 5 & -0.32 & 0.06 & -5.21 & 0.00 \\ 
   \hline
\end{tabular}

% \end{table}
% \vspace{-2em}
% \begin{singlespace}
%     \begin{footnotesize}
%     \noindent {\it Notes:} 此表展示 AIDS 模型中各項參數的估計值及其顯著性檢驗結果。如 alpha 1 的估計值為 1.237,表明該參數對模型中需求分配的影響為正,且數值較大,標準誤為 0.085,顯示估計結果穩定。所有參數的 p 值均遠小於 0.05,表示這些參數在統計上顯著,同時也可看到 AIDS 模型在解釋台灣飲料市場需求方面具有良好表現。
%     \end{footnotesize}
% \end{singlespace}

% \begin{table}[H]
%     \caption{LA/AIDS模型估計結果與顯著性} \label{laaids_coef}
%     \center
%     % latex table generated in R 4.3.1 by xtable 1.8-4 package
% Mon Nov 25 17:00:18 2024
\begin{tabular}{ccccc}
  \hline
 & Estimate & Std. Error & t value & Pr($>$$|$t$|$) \\ 
  \hline
alpha 1 & 1.16 & 0.08 & 14.70 & 0.00 \\ 
  alpha 2 & 0.42 & 0.12 & 3.60 & 0.00 \\ 
  alpha 3 & -0.27 & 0.06 & -4.29 & 0.00 \\ 
  alpha 4 & 0.76 & 0.06 & 13.59 & 0.00 \\ 
  alpha 5 & -1.07 & 0.19 & -5.59 & 0.00 \\ 
  beta 1 & -0.08 & 0.01 & -12.78 & 0.00 \\ 
  beta 2 & -0.02 & 0.01 & -1.72 & 0.09 \\ 
  beta 3 & 0.03 & 0.01 & 5.28 & 0.00 \\ 
  beta 4 & -0.07 & 0.00 & -14.29 & 0.00 \\ 
  beta 5 & 0.14 & 0.02 & 8.57 & 0.00 \\ 
  gamma 1 1 & 0.08 & 0.02 & 4.44 & 0.00 \\ 
  gamma 1 2 & -0.11 & 0.02 & -6.74 & 0.00 \\ 
  gamma 1 3 & -0.02 & 0.01 & -2.35 & 0.02 \\ 
  gamma 1 4 & 0.03 & 0.01 & 2.32 & 0.02 \\ 
  gamma 1 5 & 0.02 & 0.02 & 0.87 & 0.38 \\ 
  gamma 2 1 & -0.11 & 0.02 & -6.74 & 0.00 \\ 
  gamma 2 2 & -0.12 & 0.03 & -4.30 & 0.00 \\ 
  gamma 2 3 & -0.06 & 0.01 & -7.04 & 0.00 \\ 
  gamma 2 4 & 0.02 & 0.02 & 0.99 & 0.32 \\ 
  gamma 2 5 & 0.27 & 0.02 & 11.61 & 0.00 \\ 
  gamma 3 1 & -0.02 & 0.01 & -2.35 & 0.02 \\ 
  gamma 3 2 & -0.06 & 0.01 & -7.04 & 0.00 \\ 
  gamma 3 3 & -0.04 & 0.00 & -7.49 & 0.00 \\ 
  gamma 3 4 & 0.04 & 0.01 & 6.66 & 0.00 \\ 
  gamma 3 5 & 0.07 & 0.01 & 6.62 & 0.00 \\ 
  gamma 4 1 & 0.03 & 0.01 & 2.32 & 0.02 \\ 
  gamma 4 2 & 0.02 & 0.02 & 0.99 & 0.32 \\ 
  gamma 4 3 & 0.04 & 0.01 & 6.66 & 0.00 \\ 
  gamma 4 4 & 0.17 & 0.02 & 9.12 & 0.00 \\ 
  gamma 4 5 & -0.26 & 0.01 & -20.67 & 0.00 \\ 
  gamma 5 1 & 0.02 & 0.02 & 0.87 & 0.38 \\ 
  gamma 5 2 & 0.27 & 0.02 & 11.61 & 0.00 \\ 
  gamma 5 3 & 0.07 & 0.01 & 6.62 & 0.00 \\ 
  gamma 5 4 & -0.26 & 0.01 & -20.67 & 0.00 \\ 
  gamma 5 5 & -0.11 & 0.04 & -2.87 & 0.00 \\ 
   \hline
\end{tabular}

% \end{table}
% % \begin{singlespace}
% %     \begin{footnotesize}
% %     \noindent {\it Notes:} Standard deviations in parentheses, and standard errors in brackets. The treatment group comprises workers who underwent a mass layoff (firm reducing its employment by over 90\%), and the comparison group comprises workers who were employed at a stable firm (no more than a 30\% employment decrease) and had continuous employment during the sample period. All dollars are adjusted with CPI and displayed in 2016 NT\$ (1 NT\$ ≈ 0.033 US\$). The cumulative number of outpatient visits and cumulative medical expenses of mental illness are accumulated from the fifth to second years prior to the (pseudo) displacement. The statistics in the {\it After Matching} columns are weighted by entropy balancing (EB). The variables included in the matching process are all variables in the {\it Individual Characteristics} and {\it Firm Characteristics} panel. \\
% %     *** significant at the 1 percent level, ** significant at the 5 percent level, and * significant at the 10 percent level.
% %     \end{footnotesize}
% % \end{singlespace}
% \vspace{-2em}
% \begin{singlespace}
%     \begin{footnotesize}
%         \raggedright
%         \noindent {\it Notes:} 此表展示 AIDS 模型中各項參數的估計值及其顯著性檢驗結果。如 alpha 1 的估計值為 1.237,表明該參數對模型中需求分配的影響為正,且數值較大,標準誤為 0.085,顯示估計結果穩定。所有參數的 p 值均遠小於 0.05,表示這些參數在統計上顯著,同時也可看到AIDS 模型在解釋台灣飲料市場需求方面具有良好表現。
%     \end{footnotesize}
% \end{singlespace}

\begin{table}[H]
    \caption{AIDS模型支出彈性估計結果} \label{aids_exp}
    \center
    % latex table generated in R 4.3.1 by xtable 1.8-4 package
% Thu Nov 28 19:34:07 2024
\begin{tabular}{rr}
  \hline
 & (1) \\ 
  \hline
果蔬汁份額 & 0.52 \\ 
  碳酸飲料份額 & 0.85 \\ 
  運動飲料份額 & 1.41 \\ 
  咖啡飲料份額 & 0.54 \\ 
  茶飲料份額 & 1.34 \\ 
   \hline
\end{tabular}

\end{table}
\vspace{-2em}
\begin{singlespace}
    \begin{footnotesize}
        \raggedright
        \noindent {\it Notes:} 此表格顯示五種飲料類型的支出彈性 (Expenditure Elasticities),反映了消費者對於總支出變化的敏感度。正彈性值代表需求量隨總支出增加而增加,例如運動飲料份額的支出彈性為 1.412,表示當總支出增加 1\% 時,運動飲料的需求增加約 1.41\%。負彈性值若存在,則表示總支出增加反而減少該商品的需求。此表中大多數彈性值大於 1,意味著台灣消費者對飲料類別的需求相對敏感,特別是運動飲料和茶飲。
    \end{footnotesize}
\end{singlespace}

\begin{table}[H]
    \caption{LA/AIDS模型支出彈性估計結果} \label{laaids_exp}
    \center
    % latex table generated in R 4.3.1 by xtable 1.8-4 package
% Wed Nov 20 03:02:18 2024
\begin{tabular}{cccccc}
  \hline
 & 果蔬汁份額 & 碳酸飲料份額 & 運動飲料份額 & 咖啡飲料份額 & 茶飲料份額 \\ 
  \hline
支出彈性 & 0.55 & 0.91 & 1.38 & 0.50 & 1.32 \\ 
   \hline
\end{tabular}

\end{table}
\vspace{-2em}
\begin{singlespace}
    \begin{footnotesize}
    \noindent {\it Notes:} Standard deviations in parentheses, and standard errors in brackets. The treatment group comprises workers who underwent a mass layoff (firm reducing its employment by over 90\%), and the comparison group comprises workers who were employed at a stable firm (no more than a 30\% employment decrease) and had continuous employment during the sample period. All dollars are adjusted with CPI and displayed in 2016 NT\$ (1 NT\$ ≈ 0.033 US\$). The cumulative number of outpatient visits and cumulative medical expenses of mental illness are accumulated from the fifth to second years prior to the (pseudo) displacement. The statistics in the {\it After Matching} columns are weighted by entropy balancing (EB). The variables included in the matching process are all variables in the {\it Individual Characteristics} and {\it Firm Characteristics} panel. \\
    *** significant at the 1 percent level, ** significant at the 5 percent level, and * significant at the 10 percent level.
    \end{footnotesize}
\end{singlespace}

\begin{table}[H]
    \caption{AIDS模型 Marshallian 需求彈性估計結果}
    \center
    % latex table generated in R 4.3.1 by xtable 1.8-4 package
% Wed Nov 20 03:02:18 2024
\begin{tabular}{cccccc}
  \hline
 & 果蔬汁價格 & 碳酸飲料價格 & 運動飲料價格 & 咖啡飲料價格 & 茶飲料價格 \\ 
  \hline
果蔬汁份額 & -0.45 & -0.50 & -0.04 & 0.26 & 0.22 \\ 
  碳酸飲料份額 & -0.59 & -1.55 & -0.33 & 0.07 & 1.54 \\ 
  運動飲料份額 & -0.27 & -0.90 & -1.57 & 0.48 & 0.84 \\ 
  咖啡飲料份額 & 0.37 & 0.16 & 0.34 & 0.42 & -1.82 \\ 
  茶飲料份額 & -0.06 & 0.56 & 0.15 & -0.65 & -1.33 \\ 
   \hline
\end{tabular}
  \label{aids_marshall}
\end{table}
\vspace{-2em}
\begin{singlespace}
    \begin{footnotesize}
        \raggedright
        \noindent {\it Notes:} 此表展示 AIDS 模型中各項參數的估計值及其顯著性檢驗結果。如 alpha 1 的估計值為 1.237,表明該參數對模型中需求分配的影響為正,且數值較大,標準誤為 0.085,顯示估計結果穩定。所有參數的 p 值均遠小於 0.05,表示這些參數在統計上顯著,同時也可看到AIDS 模型在解釋台灣飲料市場需求方面具有良好表現。 
    \end{footnotesize}
\end{singlespace}

\begin{table}[H]
    \caption{LAAIDS模型 Marshallian 需求彈性估計結果}
    \center
    % latex table generated in R 4.3.1 by xtable 1.8-4 package
% Wed Nov 20 03:02:18 2024
\begin{tabular}{cccccc}
  \hline
 & 果蔬汁價格 & 碳酸飲料價格 & 運動飲料價格 & 咖啡飲料價格 & 茶飲料價格 \\ 
  \hline
果蔬汁份額 & -0.04 & -0.42 & -0.20 & 0.57 & -0.47 \\ 
  碳酸飲料份額 & -0.51 & -1.63 & -0.36 & 0.18 & 1.42 \\ 
  運動飲料份額 & -0.66 & -0.96 & -1.41 & 0.19 & 1.47 \\ 
  咖啡飲料份額 & 0.84 & 0.32 & 0.17 & 0.77 & -2.60 \\ 
  茶飲料份額 & -0.35 & 0.52 & 0.26 & -0.89 & -0.85 \\ 
   \hline
\end{tabular}
  \label{laaids_marshall}
    % \note
\end{table}
\vspace{-2em}
\begin{singlespace}
    \begin{footnotesize}
    \noindent {\it Notes:} Standard deviations in parentheses, and standard errors in brackets. The treatment group comprises workers who underwent a mass layoff (firm reducing its employment by over 90\%), and the comparison group comprises workers who were employed at a stable firm (no more than a 30\% employment decrease) and had continuous employment during the sample period. All dollars are adjusted with CPI and displayed in 2016 NT\$ (1 NT\$ ≈ 0.033 US\$). The cumulative number of outpatient visits and cumulative medical expenses of mental illness are accumulated from the fifth to second years prior to the (pseudo) displacement. The statistics in the {\it After Matching} columns are weighted by entropy balancing (EB). The variables included in the matching process are all variables in the {\it Individual Characteristics} and {\it Firm Characteristics} panel. \\
    *** significant at the 1 percent level, ** significant at the 5 percent level, and * significant at the 10 percent level.
    \end{footnotesize}
\end{singlespace}

\begin{table}[H]
    \caption{AIDS模型 Hicksian 需求彈性估計結果}
    \center
    % latex table generated in R 4.3.1 by xtable 1.8-4 package
% Wed Nov 20 03:02:18 2024
\begin{tabular}{cccccc}
  \hline
 & 果蔬汁價格 & 碳酸飲料價格 & 運動飲料價格 & 咖啡飲料價格 & 茶飲料價格 \\ 
  \hline
果蔬汁份額 & -0.35 & -0.41 & -0.00 & 0.33 & 0.44 \\ 
  碳酸飲料份額 & -0.43 & -1.40 & -0.26 & 0.18 & 1.91 \\ 
  運動飲料份額 & -0.01 & -0.65 & -1.46 & 0.67 & 1.45 \\ 
  咖啡飲料份額 & 0.48 & 0.25 & 0.38 & 0.49 & -1.59 \\ 
  茶飲料份額 & 0.19 & 0.80 & 0.25 & -0.48 & -0.76 \\ 
   \hline
\end{tabular}

\end{table}
\vspace{-2em}
\begin{singlespace}
    \begin{footnotesize}
        \raggedright
        \noindent {\it Notes:} 此表根據未補償需求 (Uncompensated Demand) 計算馬歇爾需求彈性,反映價格變動對需求的影響,並考慮了收入效應。自價格彈性:如果蔬汁的自價格彈性為 -0.452,顯示價格每上升 1\%,需求減少 0.452\%。該值大於希克斯彈性,因為馬歇爾彈性包含收入效應。交叉價格彈性:例如果蔬汁對茶飲的交叉價格彈性為 0.215,說明兩者之間的替代效應較低。相比希克斯彈性,馬歇爾彈性對政策制定更為重要,因為它包含了市場中實際的收入和價格變動對需求的綜合影響。
    \end{footnotesize}
\end{singlespace}

\begin{table}[H]
    \caption{LAAIDS模型 Hicksian 需求彈性估計結果}
    \center
    % latex table generated in R 4.3.1 by xtable 1.8-4 package
% Wed Nov 20 03:02:18 2024
\begin{tabular}{cccccc}
  \hline
 & 果蔬汁價格 & 碳酸飲料價格 & 運動飲料價格 & 咖啡飲料價格 & 茶飲料價格 \\ 
  \hline
果蔬汁份額 & 0.07 & -0.32 & -0.16 & 0.64 & -0.23 \\ 
  碳酸飲料份額 & -0.34 & -1.47 & -0.29 & 0.30 & 1.81 \\ 
  運動飲料份額 & -0.40 & -0.72 & -1.31 & 0.37 & 2.06 \\ 
  咖啡飲料份額 & 0.94 & 0.41 & 0.21 & 0.84 & -2.39 \\ 
  茶飲料份額 & -0.10 & 0.75 & 0.35 & -0.72 & -0.28 \\ 
   \hline
\end{tabular}

\end{table}
\vspace{-2em}
\begin{singlespace}
    \begin{footnotesize}
    \noindent {\it Notes:} Standard deviations in parentheses, and standard errors in brackets. The treatment group comprises workers who underwent a mass layoff (firm reducing its employment by over 90\%), and the comparison group comprises workers who were employed at a stable firm (no more than a 30\% employment decrease) and had continuous employment during the sample period. All dollars are adjusted with CPI and displayed in 2016 NT\$ (1 NT\$ ≈ 0.033 US\$). The cumulative number of outpatient visits and cumulative medical expenses of mental illness are accumulated from the fifth to second years prior to the (pseudo) displacement. The statistics in the {\it After Matching} columns are weighted by entropy balancing (EB). The variables included in the matching process are all variables in the {\it Individual Characteristics} and {\it Firm Characteristics} panel. \\
    *** significant at the 1 percent level, ** significant at the 5 percent level, and * significant at the 10 percent level.
    \end{footnotesize}
\end{singlespace}

% % latex table generated in R 4.3.1 by xtable 1.8-4 package
% Wed Nov 20 00:57:32 2024
\begin{table}[ht]
\centering
\begin{tabular}{cccc}
  \hline
Estimate & Std. Error & t value & Pr($>$$|$t$|$) \\ 
  \hline
1.24 & 0.09 & 14.51 & 0.00 \\ 
  0.54 & 0.13 & 4.17 & 0.00 \\ 
  -0.30 & 0.07 & -4.06 & 0.00 \\ 
  0.69 & 0.06 & 10.93 & 0.00 \\ 
  -1.17 & 0.22 & -5.31 & 0.00 \\ 
  -0.09 & 0.01 & -12.71 & 0.00 \\ 
  -0.03 & 0.01 & -2.43 & 0.02 \\ 
  0.03 & 0.01 & 4.84 & 0.00 \\ 
  -0.06 & 0.01 & -11.58 & 0.00 \\ 
  0.15 & 0.02 & 7.93 & 0.00 \\ 
  -0.01 & 0.02 & -0.27 & 0.79 \\ 
  -0.14 & 0.02 & -5.95 & 0.00 \\ 
  0.02 & 0.01 & 1.67 & 0.10 \\ 
  -0.02 & 0.02 & -1.24 & 0.21 \\ 
  0.15 & 0.03 & 4.96 & 0.00 \\ 
  -0.14 & 0.02 & -5.95 & 0.00 \\ 
  -0.11 & 0.04 & -3.09 & 0.00 \\ 
  -0.05 & 0.01 & -4.43 & 0.00 \\ 
  -0.01 & 0.03 & -0.31 & 0.75 \\ 
  0.31 & 0.03 & 10.17 & 0.00 \\ 
  0.02 & 0.01 & 1.67 & 0.10 \\ 
  -0.05 & 0.01 & -4.43 & 0.00 \\ 
  -0.05 & 0.01 & -6.87 & 0.00 \\ 
  0.06 & 0.01 & 7.20 & 0.00 \\ 
  0.03 & 0.01 & 1.78 & 0.08 \\ 
  -0.02 & 0.02 & -1.24 & 0.21 \\ 
  -0.01 & 0.03 & -0.31 & 0.75 \\ 
  0.06 & 0.01 & 7.20 & 0.00 \\ 
  0.14 & 0.03 & 4.61 & 0.00 \\ 
  -0.16 & 0.02 & -8.35 & 0.00 \\ 
  0.15 & 0.03 & 4.96 & 0.00 \\ 
  0.31 & 0.03 & 10.17 & 0.00 \\ 
  0.03 & 0.01 & 1.78 & 0.08 \\ 
  -0.16 & 0.02 & -8.35 & 0.00 \\ 
  -0.32 & 0.06 & -5.21 & 0.00 \\ 
   \hline
\end{tabular}
\end{table}


\newpage
% % Table generated by Excel2LaTeX from sheet 'Tab1'
% \begin{table}[htbp]
% \renewcommand{\arraystretch}{0.85}
% \setlength{\tabcolsep}{0.1mm}{}
%   \centering
%   \caption{Summary Statistics}
% {\small
%     \begin{tabular}{lccccc}
% \toprule
%           & \multicolumn{3}{c}{Before Matching} & \multicolumn{2}{c}{After Matching} \\
%           \cmidrule(r){2-4} \cmidrule(l){5-6}
%     Variable & Treatment & ~Comparison~ & Difference & ~Comparison~ & Difference \\
%           \midrule \midrule
%     \textbf{Individual Characteristics} &       &       &       &       &  \\
%     Female & 0.564 & 0.460 & 0.104*** & 0.565 & 0.000 \\
%           & (0.496) & (0.498) & [0.005] & (0.496) & [0.005] \\
%     Age at displacement   & 42.526 & 39.775 & 2.750*** & 42.512 & 0.014 \\
%           & (9.092) & (7.556) & [0.078] & (9.093) & [0.094] \\
%     Live in urban area & 0.790 & 0.737 & 0.053*** & 0.790 & 0.000 \\
%           & (0.407) & (0.440) & [0.005] & (0.408) & [0.004] \\
%     Work in urban area & 0.849 & 0.804 & 0.045*** & 0.849 & 0.000 \\
%           & (0.358) & (0.397) & [0.004] & (0.358) & [0.004] \\
%           \midrule
%     \textbf{Firm Characteristics} &       &       &       &       &  \\
%     Number of employees & 375.977 & 1,494.984 & -1,119.007*** & 411.861 & -35.884** \\
%           & (1,323.548) & (3,852.285) & [39.181] & (1,515.931) & [15.562] \\
%     Female Share & 0.491 & 0.436 & 0.055*** & 0.492 & 0.000 \\
%           & (0.212) & (0.224) & [0.002] & (0.212) & [0.002] \\
%     Average monthly wage (\$1,000) & 34.392 & 36.745 & -2.353*** & 34.432 & -0.040*** \\
%           & (13.011) & (13.518) & [0.005] & (13.027) & [0.005] \\
%     Average age & 37.530 & 36.971 & 0.559*** & 37.519 & 0.010 \\
%           & (5.364) & (5.042) & [0.052] & (5.365) & [0.055] \\
%           \midrule
%     \multicolumn{5}{l}{\textbf{Outcomes Variables in the Second Year Prior to the Displacement}}  \\
%     Real annual earnings (\$1,000) & 509.170 & 541.102 & -31.932*** & 509.638 & -0.468 \\
%           & (262.076) & (257.066) & [2.649] & (262.303) & [2.702] \\

%     Cul. \# of mental illness outpatient visits & 0.554 & 0.450 & 0.104*** & 0.551 & 0.004 \\
%           & (3.824) & (3.543) & [0.037] & (3.943) & [0.041] \\

%     Cul. medical expenses of mental illness & 0.807 & 0.585 & 0.221** & 0.708 & 0.098 \\
%          (\$1,000) & (11.344) & (9.831) & [0.102] & (10.829) & [0.112] \\

%           \midrule \midrule
%     Number of observations & 9,700 & 332,720 &       & 332,720 &  \\
% \bottomrule
%     \end{tabular}%
% }
%   \label{tab:t1}%
% \end{table}%
% \vspace{-2em}
% \begin{singlespace}
%         \begin{footnotesize}
%         		\noindent {\it Notes:} Standard deviations in parentheses, and standard errors in brackets. The treatment group comprises workers who underwent a mass layoff (firm reducing its employment by over 90\%), and the comparison group comprises workers who were employed at a stable firm (no more than a 30\% employment decrease) and had continuous employment during the sample period. All dollars are adjusted with CPI and displayed in 2016 NT\$ (1 NT\$ ≈ 0.033 US\$). The cumulative number of outpatient visits and cumulative medical expenses of mental illness are accumulated from the fifth to second years prior to the (pseudo) displacement. The statistics in the {\it After Matching} columns are weighted by entropy balancing (EB). The variables included in the matching process are all variables in the {\it Individual Characteristics} and {\it Firm Characteristics} panel. \\
% 		*** significant at the 1 percent level, ** significant at the 5 percent level, and * significant at the 10 percent level.
%         \end{footnotesize}
% \end{singlespace}


% \newpage

% % Table generated by Excel2LaTeX from sheet 'Tab2'
% \begin{table}[htbp]
% \renewcommand{\arraystretch}{1}
% \setlength{\tabcolsep}{0.3mm}{}
%   \centering
%   \caption{Long-term Impact of Job Displacement on Employment and Earnings}\label{DD_earnings}
% {\footnotesize
%     \begin{tabular}{lccccccc}
% \toprule
%           & (1)   & (2)   & (3)   & (4)   & (5)   & (6)   & (7) \\
% \midrule \midrule
%     \multicolumn{8}{l}{\textbf{Panel A: Employment}} \\  \\
%     $Disp_{i} \times \mathbf{I}[t=c+10]$ & -0.324*** & -0.326*** & -0.326*** & -0.326*** & -0.326*** & -0.326*** & -0.326*** \\
%           & (0.005) & (0.005) & (0.005) & (0.005) & (0.005) & (0.005) & (0.005) \\
%     Control Baseline Mean &       &       &       & 1.000 &       &       &  \\
%     \midrule
%     \multicolumn{8}{l}{\textbf{Panel B: Annual Earnings (1,000 NT\$)}} \\  \\
%      $Disp_{i} \times \mathbf{I}[t=c+10]$ & -302.582*** & -305.924*** & -305.922*** & -305.438*** & -305.417*** & -305.953*** & -305.988*** \\
%           & (3.828) & (3.950) & (3.950) & (3.949) & (3.951) & (4.081) & (4.083) \\
%     Control Baseline Mean &       &       &       & 509.637 &       &       &  \\
   
% \midrule \midrule
%     Observations & \multicolumn{7}{c}{5,478,720} \\
%     Basic DID & $\checkmark$ & $\checkmark$ & $\checkmark$ & $\checkmark$ & $\checkmark$ & $\checkmark$ & $\checkmark$ \\
%     Year Fixed Effect &       & $\checkmark$ & $\checkmark$ & $\checkmark$ & $\checkmark$ & $\checkmark$ & $\checkmark$ \\
%     Age   &       &       & $\checkmark$ & $\checkmark$ & $\checkmark$ & $\checkmark$ & $\checkmark$ \\
%     Individual Control &       &       &       & $\checkmark$ & $\checkmark$ &       &  \\
%     Firm Control &       &       &       &       & $\checkmark$ &       &  \\
%     Individual Fixed Effect &       &       &       &       &       & $\checkmark$ & $\checkmark$ \\
%     Unemployment Rate &       &       &       &       &       &       & $\checkmark$ \\
% \bottomrule
%     \end{tabular}%
% }
%   \label{tab:t2}%
% \end{table}%
% \vspace{-2em}
% \begin{singlespace}
%         \begin{footnotesize}
%         		\noindent {\it Notes:} This table displays the estimated coefficients of $\delta_{10}$ from equation (\ref{event}). The coefficient stands for the impact of a mass layoff in the tenth year after the displacement year ($c$). Standard errors clustered at the individual level are reported in parentheses. All regressions are weighted with EB weights. The control baseline mean is the EB-weighted mean for the comparison group in the baseline year ($t=-2$). Column (1) includes a dummy variable indicating whether an individual belongs to the treatment group (displaced worker) and the event time fixed effect. Column (2) further includes the calendar year fixed effect. Column (3) further includes the quadratic function of age. Column (4) further includes gender, birth month, wage, and county/municipality of residence in the pre-treatment period. Column (5) further includes firm characteristics (location, number of employees, average monthly wage, average age, proportion of females) in the pre-treatment period. Column (6) further includes individual fixed effects. Column (7) further includes county/municipality level unemployment rate. 
% 		 \\
% 		*** significant at the 1 percent level, ** significant at the 5 percent level, and * significant at the 10 percent level.
%         \end{footnotesize}
% \end{singlespace}
