
\section{穩健性分析}

\subsection{同質性與對稱性檢查}

本研究針對 AIDS 和 LAAIDS 模型結果進行了同質性和對稱性的檢定。同質性的檢定反映了一個基本的經濟直覺,當所有商品價格同比例變動,而消費者收入保持不變時,消費者的需求決策應維持不變。這表現為需求函數中每一行的\(\gamma_{ij}\)係數總和必須等於零(\(\sum \gamma_i = 0\))。
表\ref{aids-homo}和表\ref{laaids-homo}檢定結果顯示,無論是 AIDS 還是 LAAIDS 模型,所有行的\(\gamma_{ij}\)係數總和都精確等於零,完全符合同質性假設。

對稱性檢定則關注消費者在面對不同商品之間替代關係時的決策一致性。這反映在\(\gamma_{ij}\)係數矩陣中,交叉項應該相等\(\gamma_{ij} = \gamma_{ji}\)。
我們的研究發現,表\ref{aids-sym}和表\ref{laaids-sym}可以看到兩個模型的\(\gamma_{ij}\)係數矩陣偏差都極其接近於零,充分證實了對稱性假設的成立。

我們還進行了同質性和對稱性的綜合檢定,結果顯示,所有的綜合檢查結果均為TRUE,這意味著我們的模型完全同質性與對稱性檢定。

\subsection{似然比檢定 (Likelihood-ratio test)}

似然比檢定是一種假設檢驗,用於比較兩個模型(一個是所有參數都是自由參數的無約束模型,另一個是由線性假設約束的含較少參數的相應約束模型)的擬合度以確定哪個模型與樣本資料擬合得更好。
\( LR_{\text{stat}} \)是計算出的似然比統計量(Likelihood Ratio Statistic),其計算公式為: 
\[LR_{\text{stat}} = -2 \times (\ln L_{\text{AIDS}} - \ln L_{\text{LAAIDS}})\]
該統計量衡量了兩個模型之間的擬合效果差異。較大的統計量\( LR_{\text{stat}} \)(52.83248)表明AIDS優於LAAIDS,即增加的自由度使模型擬合得到改善。
我們針對兩者差距進行假設檢定:

\begin{itemize}
    \item \(H_0\): LAAIDS 模型足以描述數據。
    \item \(H_1\): AIDS 模型比 LAAIDS 更能有效地描述數據。
\end{itemize}
\noindent P-value是與\( LR_{\text{stat}} \)相關的顯著性水準,它反映了在原假設下觀察到這麼大的統計量的概率。虛無假設 \( H_0 \) 為LAAIDS模型成立,即約束條件下擬合效果更好。因為P-value 很小( \( 3.63265 \times 10^{-13} \)),遠遠低於0.01,表示在1\%的信心水準下有顯著差異,所以拒絕虛無假設。這意味著非受限模型(AIDS 模型)顯著優於受限模型(LAAIDS 模型),即沒有線性約束的模型擬合效果更好。
