\section{研究結果}\label{dis_labor_health}

igure \ref{eventstudy} presents our estimated dynamic effects of displaceme

\subsection{使用AIDS模型分析}
Figures \ref{emp_e} and \ref{earnings} show that the employment and annual earnings decline sharply in the year following displacement and recover limitedly ten years after displacement. Specifically, we note approximately a 30\% decrease in the probability of employment and a 40\% reduction in the annual earnings, ten years after the year of displacement. Consistent with graphical evidence on labor market outcomes, Figure \ref{mental} suggests that compared to the control group, displaced workers had higher utilization of medical services due to mental illness. Importantly, prior to layoff, these outcomes of the displaced workers align closely with those of the non-displaced counterparts, suggesting that their post-displacement differences are not driven by differential pre-trends between treatment and control groups. 

\subsection{使用LAAIDS模型分析}\label{dis_labor}

Figure \ref{eventstudy} presents our estimated dynamic effects of displacement on earnings and employment from model (\ref{event}). Before the reference year (two years before displacement), earnings for workers who will be displaced and non-displaced workers follow a similar trend. In the year prior to displacement, there was a significant decline in earnings by roughly 2,000 NTD, economically small compared to previous years.\footnote{\citet{lachowska} and \citet{schmieder} also found a significant earning loss prior to displacement using data from the U.S. and Germany.} Annual earnings for displaced workers drop by around 270,000 NTD in the year of displacement and 340,000 NTD in the year following displacement (about 67\%). While there is a small recovery in annual earnings two years after displacement, a substantial long-term effect is still visible ten years after a mass layoff. Similarly, Figure \ref{eventstudy} (a) shows the probability of employment mirrors the earnings trend, with a sharp initial decline (about 40\%) in the year after displacement and a partial recovery thereafter. After ten years, the employment probability remains nearly 30\% lower than before the displacement. 
