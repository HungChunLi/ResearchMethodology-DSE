
\section{AIDS及LAAIDS模型參數估計結果}

% \subsection{係數解釋}

% 表 \ref{coef} 記錄了AIDS與 LAAIDS模型的估計結果,其探討了果蔬汁、碳酸飲料、運動飲料、咖啡飲料及茶飲料的需求特徵。
% \texttt{alpha} 係數反映了收入不變時,各飲料的基礎需求水準。兩模型的結果一致指出,果蔬汁和咖啡飲料擁有較高的基礎需求。在AIDS 模型及LAAIDS 模型中,果蔬汁的\texttt{alpha 1}估計值分別為 1.24和 1.16;咖啡飲料的\texttt{alpha 4}估計值則為 0.69和 0.76,顯示這兩類飲料對消費者來說是穩定的日常選擇,特別是果蔬汁的健康屬性和咖啡飲料的功能性使其市場地位穩固。相較之下,碳酸飲料的基礎需求偏低,\texttt{alpha 2} 值分別為 0.54和 0.42,顯示健康意識的提升對其需求造成了一定影響,運動飲料的需求基礎有限,\texttt{alpha 3}值在兩模型中均為負(-0.30 和 -0.27),說明其消費群體較為特定,需求具有較高彈性。而茶飲料的\texttt{alpha 5}值為-1.17 和 -1.07,反映其基礎需求在市場中相對最低。

% beta 係數衡量總支出變化對需求的影響。\texttt{beta 1}(果蔬汁)的 beta 值在兩模型中均為負,顯示總支出增加時其需求略有下降,可能因為消費者轉向更高端飲品如運動飲料和茶飲料。\texttt{beta 3}(運動飲料)的值在兩模型中均為顯著正值,表明其需求隨總支出上升而增加。同樣,\texttt{beta 5}(茶飲料)的 beta 值亦為正,顯示總支出增加會促進需求增長,強化了其作為正常財的特性。\texttt{beta 2}(碳酸飲料)的值雖接近顯著性,但在兩模型中呈現出正值,表明總支出對其需求影響有限但可能略有促進。而\texttt{beta 4}(咖啡飲料)的值為負,顯示其需求會隨總支出提升而減少。

% gamma 係數揭示了飲料間的替代與互補關係。兩模型均表明,\texttt{gamma 5 4}(茶飲料與咖啡飲料)之間的替代效應較弱,當茶價格上升時,咖啡需求也會減少(gamma 值為負),可能反映了兩者目標消費群體的差異。\texttt{gamma 2 5}(碳酸飲料與茶飲料)之間的替代性較強,當茶價格上升時,碳酸飲料需求增加,表明碳酸飲料是茶的重要替代品。此外,果蔬汁與其他飲料的 gamma 值表現出一定的互補性。


\subsection{支出彈性}

表 \ref{exp} 記錄了 AIDS 和 LAAIDS 模型的支出彈性結果,探討果蔬汁、碳酸飲料、運動飲料、咖啡飲料及茶飲料在台灣市場的需求在面對總支出變動時的敏感性。

運動飲料在AIDS 模型和LAAIDS 模型中的支出彈性分別為 1.41與 1.38,為所有飲料類型中最高,顯示其需求對總支出變動極為敏感。茶飲料的支出彈性分別為 1.34與 1.32,隨著總支出的提升,茶飲料需求大幅增長。果蔬汁的支出彈性在兩模型中分別為 0.52和 0.55,顯示其需求對總支出變動不敏感,這表明果蔬汁的消費穩定,即使總支出波動,其需求變化幅度也較小。咖啡飲料的支出彈性在兩模型中分別為 0.54與 0.50,標示消費者對咖啡飲料的需求穩定。碳酸飲料的支出彈性在兩模型中分為 0.85與 0.91,接近 1,顯示其需求隨總支出的變化呈接近比例的增長。


\subsection{Marshallian 需求彈性}

Marshallian 需求函數又稱為未補償需求函數(uncompensated demand function),描述的是在給定收入和價格條件下,消費者如何選擇商品組合以最大化效用,其考慮了包含收入在內的所有外生因素,因此顯示各飲料需求對於價格變動的總反應。從表 \ref{aids_marshall}和表\ref{laaids_marshall}可以看出對各種商品估計的 Marshallian 需求彈性,揭示其對自身與其他飲料價格變動的敏感性及市場定位。

果蔬汁在 AIDS 模型及 LAAIDS 模型中的自價格彈性分別為 -0.45 及 -0.04,顯示其需求穩定,碳酸飲料自價格彈性分別為 -0.59 及 -0.51 ,表明需求對價格波動較敏感,健康意識提升可能進一步影響其市場。運動飲料自價格彈性為 -0.27 及 -0.66 ,反映其需求在健康飲品市場中對價格有所敏感。咖啡飲料的自價格彈性為 0.37 及 0.84 ,可能反映品牌價值帶來的特殊消費行為。茶飲料自價格彈性在兩模型中分別為 -0.06  及 -0.35 ,顯示其需求對價格變動影響最小,具有高度穩定性。

在交叉價格彈性中,果蔬汁與碳酸飲料需求表現出一定的互補性,在 AIDS 模型及 LAAIDS 模型中分別為 -0.50 及 -0.42;而果蔬汁與運動飲料之間則顯示出顯著的替代效應,在兩模型中分別為 0.84 及 1.47。咖啡飲料與茶飲料之間的互補效應在兩模型中均顯著,在兩模型中分別為-1.82 及 -2.60,當茶價格上升時,咖啡需求顯著下降,顯示兩者在功能性消費中具有緊密聯動。


\subsection{Hicksian 需求彈性}

Hicksian 需求函數又稱為受補償需求函數(compensated demand function),描述的是在給定效用水準下,消費者如何選擇商品組合以最小化支出。在 Hicksian 需求中,需求彈性消除了收入效果,只考慮了替代效果。從表\ref{aids_hicks}和表\ref{aids_hicks}中可以看到 AIDS 和 LAAIDS 模型對各商品 Hicksian 需求彈性的估計結果。

在 AIDS 和 LAAIDS 模型中,碳酸飲料和運動飲料均表現出極高的需求彈性,碳酸飲料的自彈性分別為 -1.398 和 -1.466,而運動飲料的自彈性則為 -1.463 和 -1.307。這些數據顯示,這兩類飲料的需求對價格變動非常敏感,價格上升 1\% 會導致其支出占比顯著下降,表示消費者在價格波動時極易轉向其他替代商品。在 AIDS 模型中,果菜汁的價格彈性為 -0.354,而在 LAAIDS 模型中則為 0.069,表示果菜汁的需求對價格的敏感性相對較低。

茶飲料與碳酸飲料之間的替代效應最為強烈,當茶飲料價格上升 1\% 時,碳酸飲料的支出占比在 AIDS 模型中增加了 1.909,而在 LAAIDS 模型中增加了 1.806。同樣,茶飲料與運動飲料之間的替代效應也非常顯著,茶飲料價格每上升 1\%,運動飲料的支出占比增加 1.449 至 2.058。互補效應則在茶飲料與咖啡飲料之間顯得尤為顯著,當茶飲料價格上升 1\% 時,咖啡飲料的支出占比下降了 1.594 至 2.389,這表明消費者經常會同時購買這兩種飲料。

