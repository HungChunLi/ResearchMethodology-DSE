% \section{資料與敘述統計}\label{data}

% \subsection{資料}
% Our study employs administrative data acquired from Taiwan's Health and Welfare Data Science Center (HWDC) to carry out the empirical analysis. We specifically make use of four distinct data sources from the HWDC: (1) National Health Insurance (NHI) enrollment records; (2) NHI claim files for outpatient care; (3) NHI claim files for inpatient care. To connect individuals across these data sources, we utilize their scrambled national identification numbers.


% \subsection{敘述統計}\label{em_strategy}
% Our sample comprises individuals aged 20-65 who were employed in a firm with at least 5 employees in the baseline year. We define displaced workers, our treatment group, as those who were employed for a minimum of five years prior to losing their jobs and underwent a mass layoff in a given year between 2005 and 2007. A mass layoff is characterized by a firm reducing its employment by over 90\% compared to the previous year. We track these workers for 16 years, including five years before and ten years after job loss. The control group consisted of non-displaced workers who had continuous employment during the sample period and worked in stable firms, which had no more than a 30\% employment decrease in December year-on-year.\footnote{Note that they are not necessary to stay in the same firm.} Since many workers satisfy this criterion, we randomly select 10\% of them to serve as the control group. Our final sample is resulting in 9,700 workers from the treatment group and 332,720 from the control group.

% Our empirical strategy to identify the dynamic effects of displacement on earnings and mental health involves estimating the following regression:
% \begin{equation}\label{event}
% 	Y_{it}= \underset{k = -5}{\overset{10}{\sum }}\delta_{k} \cdot Disp_{i} \times \mathbf{I}[t=c+k] + \underset{k = -5}{\overset{10}{\sum }}\gamma_{k} \cdot \mathbf{I}[t=c+k] 
% 	+ \alpha_i 
% 	+ \pi_t
% 	+ X_{it} \beta
% 	+\varepsilon_{it}.
% \end{equation}
% $Y_{it}$ is the outcome of interest for worker $i$ in year $t$: 1) Employment (a dummy indicating working for at least one month per year or not); 2) Annual earnings; 3) Cumulative number of visits for mental illness; 4) Cumulative medical expenses of mental illness (including both outpatient and inpatient care).   $Disp_{i}$ indicates whether worker $i$ is a displaced worker. $I(t=c+k)$ is a dummy variable indicating $k$ years after (pseudo) mass layoff year, $c$, where $k$ does not include $-2$ since we consider two years prior to a job loss as the reference year.\footnote{For the comparison group, the pseudo mass layoff year is the year we use to confirm the firm did {\it not} go through a mass layoff and remain no more than 30\% reduction in the firm size.)} $\gamma_{k}$ represents the evolution of outcomes among non-displaced workers. $\delta_{k}$ is the coefficient of interest, which measures the change in outcomes among displaced workers with respect to the reference year ($k=-2$), relative to the change of non-displaced workers. We additionally control individual fixed effects ($\alpha_i$) and calendar year fixed effects ($\pi_t$). $X_{it}$ are other controls, mainly quartic function of worker's age and county/municipality level unemployment rate.

\section{資料蒐集與處理}

\subsection{資料蒐集}
本研究使用的資料主要來自於「經濟部工業產銷存動態調查資料庫」\citep{moea_statistics},涵蓋5種飲料類別(果蔬汁飲料\footnote{含天然果汁/蔬菜汁或還原果汁/蔬菜汁10\%以上,直接供飲用之果汁/蔬菜汁飲料。例如稀釋果蔬汁、清淡果蔬汁、發酵果蔬汁、稀釋發酵果蔬汁、清淡發酵果蔬汁、果肉飲料。
}、碳酸飲料\footnote{在除去鹵素飲用水中加壓,添加二氧化碳及果實香料、果汁;或可樂子實葉抽出液;或Saraparilla根抽出液等調味料之碳酸飲料。例如汽水、可樂、沙士。
}、運動飲料\footnote{具可調解人體電解質功能之飲料,調整為等張滲透壓,以便自人體腸道迅速吸收,PH值在2.5~3.8之間,電解質濃度(ug/ml)則分別為鈉離子552以下、鎂離子24以下、鉀離子195以下、氯離子639以下、鈣離子60以下、磷酸根離子190以下。
}、咖啡飲料\footnote{利用咖啡粉或咖啡豆研磨、浸泡、萃取、調理,添加奶精、糖水或調味料之飲料,其咖啡因若超過200ppm則需標示,但不得超過500ppm。例如純咖啡飲料、調味咖啡飲料。
}及茶類飲料\footnote{利用茶葉或茶葉梗浸泡、萃取、調理,添加糖水或調味料之飲料,其咖啡因若超過200ppm則需標示,但不得超過500ppm。例如烏龍茶、花茶、紅茶、綠茶、調味茶(如檸檬茶)。
})的銷售量與銷售值的月資料,詳細記錄了每種飲料的市場表現。
% 此外,另一部分資料來自於「勞動部勞動統計查詢網」的國民所得月度統計資料,用以反映消費者的收入水準。
所有資料的涵蓋期間為1982年至2024年,共收集到547筆月統計資料,確保了樣本的時間跨度,
% 這些數據不僅提供了每類飲料的銷售量和銷售值,也包含了與消費行為密切相關的消費者所得,
為後續的DSE(Demand System Estimation)分析提供了關鍵依據。

\subsection{資料處理}

在完成資料蒐集後,我們按步驟進行了資料處理,目的是提高數據的可靠性與一致性。首先,我們使用 R 統計軟體對資料進行清理,包括合併不同來源的數據、去除重複資料、以及處理遺漏值等工作。對於部分遺漏值,考量到補值可能帶來的偏誤,我們選擇將無法合理填補的觀察值移除。此外,由於不同飲料類別的統計起始時間不一致,我們將分析的起始時間進行統一,確保資料具有可比性。為了進一步豐富研究變數,我們還根據銷售量與銷售值兩個變數計算並新增了每種飲料的單位價格,為我們後續使用模型估計去衡量價格對消費者需求的影響提供了必要的解釋變數。

\subsection{敘述統計}

表 \ref{tab:summary_statistics} 呈現了不同飲料類別在銷量、銷售值、價格方面的敘述統計結果。整體來看,茶的銷量與銷售值遠高於其他飲料類別,分別達到平均值 79,898.6 千元和 1,421,873.2 千元,顯示其市場需求和佔有率明顯高於其他飲料,可能為企業的主要收入來源。
此外,運動飲料的銷量和銷售值相對較低,但其平均價格為 23.14 元,位居所有飲料中的第二高,反映其市場定位可能側重於高單價、高附加值的策略,而非以銷量驅動收益。
價格方面,咖啡以平均價格 38.34 元位居所有飲料之首,顯示出其作為高端消費品的市場定位,與碳酸飲料 19.74 元的低價形成對比,後者更適合作為大眾化產品。
值得注意的是,運動飲料的價格標準差達到 3.82 元,顯示其價格變動幅度較大,可能受不同品牌或包裝策略的影響。
整體而言,各飲料類別的市場需求與價格策略呈現出多樣化特徵,其中茶憑藉高需求和穩定性成為市場領導產品,而咖啡和運動飲料則依靠其高單價策略實現利潤最大化。
這些結果為後續分析不同飲料市場的定位策略與消費趨勢提供了有價值的參考。


\subsection{單根檢定 Augmented Dickey-Fuller (ADF) Test}

在使用 AIDS 模型進行估計前,確認資料是否為定態是關鍵步驟。若資料為非定態,可能導致參數估計失準,檢定結果不可靠。AIDS 模型假設需求系統達到穩定均衡,而非定態資料可能反映消費行為或價格水準的結構性變化,例如季節性或長期趨勢變化,因而違背模型假設。在圖\ref{trend}中,我們可以觀察到銷售量疑似存在季節性及長期趨勢變化,因此本節透過 Augmented Dickey-Fuller (ADF) Test 進行單根檢定,確認資料是否為定態。這不僅能提升估計的準確性,也能確保需求彈性估計的可靠性與穩健性。

在 ADF 檢定中,我們擬定以下的虛無假設及對立假設:
\begin{itemize}
    \item \(H_0\): 「該序列存在單根,即非定態。」
    \item \(H_1\): 「該序列不存在單根,即定態。」
\end{itemize}
根據表 \ref{tab:stationarity} 中的檢定統計量及 p 值,我們可以判斷不同變數的定態性。當 p 值小於顯著水平 0.05 時,表示我們可以拒絕虛無假設,從而判定該序列為定態。結果顯示,果蔬汁銷量(ADF = -5.220, p = 0.01)、碳酸飲料銷量(ADF = -11.645, p = 0.01)、運動飲料銷量(ADF = -4.653, p = 0.01)以及果蔬汁銷售值和碳酸飲料銷售值的 p 值均小於 0.05,顯示這些變數在時間序列上為定態,意味著它們在樣本期間中沒有顯著的趨勢或結構性變化。

相較之下,咖啡銷量(ADF = -3.344, p = 0.064)、茶銷量、果蔬汁價格(ADF = -3.368, p = 0.059)、碳酸飲料價格、運動飲料價格及茶價格的 p 值均大於 0.05,顯示這些變數為非定態,時間序列統計特性可能隨時間變化。特別是,茶銷量和果蔬汁價格的 p 值略高於顯著性水平,表明它們無法拒絕存在單位根的假設。在後續分析中,對於這些非定態變數,可能需要進一步進行變換(如差分處理)以達到定態,確保模型估計的穩健性。