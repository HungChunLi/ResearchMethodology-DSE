\begin{abstract}
    本研究目的在於透過分析不同飲料的需求彈性與價格敏感度,了解市場中商品間的替代與互補關係,並對健康趨勢的影響進行探討。
    本研究使用了「經濟部工業產銷存動態調查資料庫」的 1982 至 2024 年月度數據,涵蓋果菜汁、碳酸飲料、運動飲料、咖啡飲料與茶飲料等五種類型飲料的銷售量與銷售值,基於 AIDS 和 LAAIDS 模型,分析台灣飲料市場中五種飲料的需求結構,估算支出彈性、自身價格彈性與交叉價格彈性。
    分析結果顯示,碳酸飲料和運動飲料對價格高度敏感,其需求彈性顯著高於其他飲料,反映出健康趨勢對其消費行為的影響。
    相比之下,果菜汁需求穩定,彈性較低,顯示其作為健康飲食組成的重要性。
    %交叉價格彈性結果顯示,茶飲料與碳酸飲料之間具有顯著的替代效應,而茶飲料與咖啡飲料之間則存在強互補效應。
    未來可進一步研究健康資訊和政策對需求結構的影響,為市場策略和政策制定提供參考。
    \footnote{本文所有資料與程式碼開源於 \url{https://github.com/HungChunLi/ResearchMethodology-DSE},供讀者下載重製。}
    \footnote{字數統計:}
\end{abstract}
\vspace{-0.7cm}
\begin{keywords}
    台灣飲料市場、需求系統分析、AIDS 和 LAAIDS 模型、健康趨勢
\end{keywords}
\vspace{-0.7cm}
\renewcommand{\abstractname}{文章重點} % 暫時修改摘要標題
\begin{abstract}
    \vspace{-7ex} % 與正文間距
    \noindent 
    \begin{itemize}
        \item[1] 基於 AIDS 和 LAAIDS 模型,分析台灣飲料市場中五種飲料的需求結構。
        \item[2] 碳酸飲料和運動飲料對價格敏感,需求彈性高,反映健康趨勢對其消費行為的影響。
        \item[3] 果菜汁需求穩定,彈性較低,顯示其作為健康飲食組成的重要性。 
    \end{itemize}
\end{abstract}