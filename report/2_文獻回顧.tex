\section{文獻回顧}

在探討台灣飲料市場的需求系統時,我們主要聚焦於消費者健康意識的增強,以及不同產品價格對於消費者偏好的影響。針對美國飲料市場,\citet{RN2} 的研究運用「近似理想需求系統」(AIDS)模型分析了不同飲料的需求彈性,揭示了各類產品在奢侈性和必需性上的特徵。研究發現,非碳酸飲料的支出彈性較高,因此被視為奢侈品,而咖啡和茶則顯示其作為必需品的屬性。這表明,消費者在價格和支出上對不同飲料的需求具有顯著的敏感性差異。

\citet{RN1} 在日本市場的研究中,運用了 LA/QUAIDS 模型,分析健康標籤和功能性成分對消費者偏好的影響。該研究指出,不同年齡層的消費者對飲料的偏好存在顯著差異:年輕人更傾向於選擇果汁和牛奶,而老年人則偏好茶飲。此外,溫度對飲料需求的影響也十分顯著,隨著氣溫上升,冷飲的需求增加,而熱飲需求則有所下降。這些結果展示了人口統計因素與季節性變化在需求系統中的重要性,對於理解台灣市場中不同飲料在不同氣候條件下的需求特徵具有參考價值。

\citet{RN9} 的研究表明,飲料製造商正逐步響應消費者對健康產品需求的變化,通過減少產品中的糖、鈉及人工甜味劑,並引入「低糖」、「低鈉」等健康標籤來吸引消費者。與此同時,\citet{RN3} 的文獻回顧聚焦於歐洲地區軟性飲料(soft drinks)的消費模式,結合各國代表性飲食調查數據,探討了健康意識增強、政策干預及人口結構對飲料需求的影響。這些研究為健康政策的制定及市場需求的精準分析提供了實證支持。

此外,\citet{Rn15} 提出了一個全面的消費行為模型,揭示了健康認知、社會影響和媒體資訊在驅動功能性飲料需求中的核心作用。該研究指出,消費者在面臨健康威脅時,對具有增強免疫力、抗氧化或促進整體健康功能的飲料表現出更高的需求彈性。

隨著消費者對含糖飲料(SSBs)健康風險的認知不斷加深,市場對低糖或無糖飲料的需求也呈現出顯著的增長趨勢,尤其在青少年和老年消費者群體中更為明顯\citep{RN3}。隨著健康意識的普及,飲料行業逐步向生產更健康的產品轉型,我們希望通過需求系統模型驗證這一趨勢是否同樣適用於台灣市場,並進一步探索台灣消費者對無糖和低糖飲料需求的潛在增長,以評估健康意識增強對需求的具體影響。
 