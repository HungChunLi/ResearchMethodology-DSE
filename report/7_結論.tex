\section{結論} \label{conclusion}

本研究基於 AIDS (Almost Ideal Demand System)和 LAAIDS (Linear Approximate Almost Ideal Demand System) 模型,分析了台灣飲料市場中五種類型飲料的需求結構。透過模型結果,我們估算了各類飲料的支出彈性、自身價格彈性與交叉價格彈性,並揭示了商品之間的替代與互補關係,同時辨識出潛在的健康相關趨勢。

研究結果顯示,碳酸飲料和運動飲料的需求對自身價格的敏感性極高。
這表明兩類飲料屬於價格高度敏感的商品,消費者對價格上漲反應強烈,結合健康趨勢來看,碳酸飲料的高需求彈性可能反映了消費者對其高糖、高熱量特性的顧慮逐漸增強,價格上漲進一步促使消費者選擇替代品。

與之對比,果菜汁需求穩定,顯示其需求彈性較低%(AIDS:-0.354;LAAIDS:0.069)
,表明消費者將其視為日常飲食的一部分,甚至可能將其作為健康飲食的組成,即使價格上升,需求量也幾乎不受影響。

交叉價格彈性分析中,茶飲料與碳酸飲料之間的替代效應最為顯著%(AIDS:1.909;LAAIDS:1.806)
表明消費者會在這兩者之間靈活轉換。此外,茶飲料與咖啡飲料之間的強互補效應%(AIDS:-1.594;LAAIDS:-2.389)
表明,這兩種飲料可能因其健康屬性而經常被搭配購買。

未來研究應進一步探索健康相關資訊的影響,例如將健康屬性指數或相關政策納入模型,以量化其對需求結構的改變。同時,考慮將個體消費者數據與市場層面數據結合,提供更具代表性的需求洞察。
