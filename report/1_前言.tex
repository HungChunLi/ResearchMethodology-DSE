\section{前言}

\subsection{研究背景}

隨著健康意識的逐漸提升,飲料市場的需求格局正在發生顯著變化。台灣作為一個消費者多樣化的市場,飲料需求受價格、收入及健康相關因素影響的特性具有研究價值。本研究以台灣飲料市場為例,探討消費者行為與需求系統的關聯,尤其關注健康意識的增強對不同飲料類別需求的影響。
過去的相關研究為本研究提供了堅實的理論基礎。\citet{RN2} 以「近似理想需求系統」(AIDS)模型分析美國市場不同飲料的需求彈性,發現非碳酸飲料具有奢侈品屬性,而咖啡與茶則被視為必需品。
\citet{RN1} 採用 LA/QUAIDS 模型分析日本市場,揭示了消費者年齡層與季節性變化對飲料需求的顯著影響。
同時,\citet{RN9} 指出,飲料製造商正響應健康需求,減少糖、鈉等成分,並引入健康標籤來吸引消費者。
此外,\citet{RN15} 強調了健康認知與媒體資訊在推動功能性飲料需求中的核心作用。

\subsection{研究目的}

為進一步驗證健康意識對台灣飲料市場的影響,本研究基於台灣經濟部工業產銷存動態調查資料庫及勞動部勞動統計查詢網,蒐集了1982年至2024年間五大飲料類別(果蔬汁飲料、碳酸飲料、運動飲料、咖啡飲料及茶類飲料)的月度銷售數據及收入數據。透過 AIDS 和 LA/AIDS 模型的應用,我們將探索健康意識提升如何影響台灣市場無糖與低糖飲料的需求,並評估價格與支出變化對飲料需求的影響特徵。
本研究期望填補台灣飲料市場健康需求相關研究的空白,為產業策略規劃與政策制定提供實證支持。