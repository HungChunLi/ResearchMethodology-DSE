\section{前言}

\subsection{研究背景}

隨著全球健康意識的提升,消費者對飲料的選擇正發生顯著變化。在台灣,這一趨勢尤為明顯。根據財政部統計,台灣飲料業者家數從2017年的21,346家增至2021年的27,414家,年均成長率達6.4\%,顯示市場持續擴張。
在台灣,手搖飲料市場的蓬勃發展反映了消費者對多樣化飲品的追求。根據\citet{lndata2022taiwandrink}的分析,台灣手搖飲料市場呈現多樣性選擇,吸引各年齡層消費者,顯示消費者對口感和品味的無窮追求。
然而,健康意識的提升可能正在改變這一市場的需求格局,消費者對健康的重視,正引導市場從傳統高糖飲料轉向無糖、低糖及功能性飲品\citep{RN3}。

% 過去的研究為我們理解這一轉變提供了基礎。\citet{RN2}利用「近似理想需求系統」(AIDS)模型分析美國市場,發現非碳酸飲料被視為奢侈品,而咖啡與茶則為必需品。\citet{RN1}採用LA/QUAIDS模型研究日本市場,揭示消費者年齡層與季節性對飲料需求的顯著影響。此外,\citet{RN9}指出,飲料製造商正響應健康需求,減少糖、鈉等成分,並引入健康標籤以吸引消費者。\citet{RN15}則強調健康認知與媒體資訊在推動功能性飲料需求中的核心作用。


\subsection{研究目的}

為深入探討健康意識對台灣飲料市場的影響,本研究
%將基於「經濟部工業產銷存動態調查資料庫」\citep{moea_statistics},蒐集1982年至2024年間果蔬汁飲料、碳酸飲料、運動飲料、咖啡飲料及茶類飲料的月度銷售及收入數據。
透過應用AIDS和LAAIDS模型,分析健康意識提升如何影響無糖與低糖飲料的需求,並評估價格與支出變化對飲料需求的特徵。