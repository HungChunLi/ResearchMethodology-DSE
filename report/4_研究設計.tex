\section{研究設計}

本文分別採用 AIDS (Almost Ideal Demand System)和 LA/AIDS (Linear Approximate AIDS) 兩種需求系統模型進行分析。
兩種方法皆用於分析多個商品的需求及其需求彈性,主要差別在於價格指數的處理方式,。
以下兩節將分別敘述 AIDS 及 LA/AIDS 的模型架構。

\subsection*{需求函數}
 模型中各商品的的需求函數為:
\begin{equation}
w_i = \alpha_i + \sum_{j=1}^{5} \gamma_{ij} \ln(P_j) + \beta_i \ln\left(\frac{X}{P}\right),
\end{equation}
其中,\(w_i\) 表示第 \(i\) 類飲料的支出比例,定義為該類飲料的支出占總支出的比例,即:
\begin{equation}
w_i = \frac{P_i Q_i}{X}.
\end{equation}
在這裡,\(P_i\) 是第 \(i\) 類飲料的價格,\(Q_i\) 是該類飲料的消費數量,而 \(X = \sum_{i=1}^5 P_i Q_i\) 是所有飲料的總支出,可能會隨月收入的變化而改變。此外,\(\ln(P_j)\) 是第 \(j\) 類飲料價格的自然對數,用於反映價格變化對需求的影響。
模型的待估參數包括 \(\alpha_i\)、\(\gamma_{ij}\) 和 \(\beta_i\),分別具有以下意義:
\begin{itemize}
    \item \(\alpha_i\):基礎支出比例,表示在其他條件不變時,第 \(i\) 類飲料的消費佔比。
    \item \(\gamma_{ij}\):描述第 \(j\) 類飲料價格對第 \(i\) 類飲料支出的影響。
    \item \(\beta_i\):支出彈性,反映總支出變動對第 \(i\) 類飲料需求的影響。
\end{itemize}


\subsection*{價格指數}
價格指數 \(P\) 用於調整總支出的影響。在AIDS模型中,其非線性表達式為:
\begin{equation}
\ln(P) = \alpha_0 + \sum_{j=1}^{5} \alpha_j \ln(P_j) + \frac{1}{2} \sum_{j=1}^{5} \sum_{k=1}^{5} \gamma_{jk} \ln(P_j) \ln(P_k),
\end{equation}
其中,\(\alpha_0\) 是基準常數,用於表示價格指數的基本水平;\(\alpha_j\) 是第 \(j\) 類飲料價格的影響係數;\(\gamma_{jk}\) 是第 \(j\) 和第 \(k\) 類飲料價格的交叉效應,用於衡量價格互動對需求的影響。
由於價格指數 \(P\) 的非線性形式較難直接處理,在 LA/AIDS 模型中,通常會選擇線性近似方法來簡化價格指數的計算,例如使用 Stone 指數:
\begin{equation}
	\ln(P) \approx \ln(P) = \sum_{j=1}^{5} w_j \ln(P_j).
\end{equation}

\subsection*{彈性}
AIDS 模型允許我們計算三種類型的需求彈性:
\begin{itemize}
    \item \textbf{支出彈性} (\(\eta_i\)):
    \begin{equation}
    \eta_i = 1 + \frac{\beta_i}{w_i},
    \end{equation}
    該彈性表示總支出變化對第 \(i\) 類飲料需求的影響。
    % 當 \(\eta_i > 1\) 時,該飲料為奢侈品;當 \(\eta_i < 1\) 時,則為必需品。
    \item \textbf{自價格彈性} (\(\varepsilon_{ii}\)):
    \begin{equation}
    \varepsilon_{ii} = -1 + \frac{\gamma_{ii}}{w_i} - \beta_i \ln(X / P),
    \end{equation}
    該彈性衡量第 \(i\) 類飲料價格變化對其自身需求的影響,通常為負值。
    \item \textbf{交叉價格彈性} (\(\varepsilon_{ij}\)):
    \begin{equation}
    \varepsilon_{ij} = \frac{\gamma_{ij}}{w_i} - \beta_i \ln(X / P),
    \end{equation}
    該彈性衡量第 \(j\) 類飲料價格變化對第 \(i\) 類飲料需求的影響。若 \(\varepsilon_{ij} > 0\),則說明兩者為替代品;若 \(\varepsilon_{ij} < 0\),則為互補品。
\end{itemize}

總結來說,AIDS 模型的核心特徵在於其價格指數的非線性特性,使其能夠精確捕捉多商品之間的需求互動與價格影響。通過支出彈性、自價格彈性和交叉價格彈性的計算,該模型能夠有效分析商品之間的需求關係,並為市場需求預測和政策評估提供有力的工具。
